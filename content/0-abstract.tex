% !TEX root = ../my-thesis.tex
%
\pdfbookmark[0]{Abstract}{Abstract}
\chapter*{Abstract}
\label{Abstract}
\vspace*{-10mm}

Questa ricerca si basa sull'intuizione che i risultati raggiunti nell'ambito della \textit{Knowledge Base Population} mediante \textit{Distant Supervision}, gli sviluppi introdotti dalla nuova tecnica di \textit{Data Programming} e l'efficienza dimostrata dalle \textit{Bidirectional Long Short Term Memory Network} (\textit{Bi-LSTM}) con input sequenziali come il testo libero, possano essere combinati per ottenere risultati allo stato dell'arte nella costruzione di basi di conoscenza \textit{Linked Data}. 

Il sistema software sviluppato fornisce una dimostrazione empirica di come una base di conoscenza \textit{Linked Data} come \textit{DBpedia} possa essere popolata con nuove triple, composte da predicati già definiti in un'ontologia \textit{OWL} (\textit{DBpedia Ontology}), analizzando testi enciclopedici scritti in linguaggio naturale (\textit{Wikipedia}). L'estrazione di tali informazioni avviene utilizzando tecniche di \textit{Natural Language Processing} (\textit{NLP}) e una rete neurale di tipo Bi-LSTM. In mancanza di dati naturalmente etichettati, si genera l'insieme di allenamento per il classificatore in maniera automatizzata secondo il paradigma di \textit{Data Programming} implementato nel framework \textit{Snorkel}.

Lo scopo del lavoro è quello di valutare il progetto sviluppato analizzando l'efficacia delle tecniche appena citate e confrontare i risultati con quelli riportati per i lavori allo stato dell'arte nella letteratura di competenza per la costruzione di basi di conoscenza.

 








